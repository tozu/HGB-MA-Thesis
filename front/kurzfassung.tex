\chapter{Kurzfassung}

\begin{german}
In vergangen Jahre etablierte sich die Automatisierung als ein wachsender Trend und rief, insbesondere im Bereich der Automobilindustrie, die Entwicklung von Fahrassistenz-Systemen und selbstfahrenden Fahrzeugen hervor. Einhergehend mit der Entwicklung solcher Lösungen muss auch das Bewusstsein und die Vorsicht gegenüber den beteiligten Verkehrsteilnehmer berücksichtigt werden.

Daher ist es das Ziel dieser Thesis ein Prototyp eines \textit{``Fußgänger-Warnsystems''} zu entwickeln, welches die Erkennung von in der Nähe befindlichen Fußgängern ermöglicht und an die straßenseitigen Infrastruktur (RSU) weiterleitet.
Bei Empfang der Standortinformation wertet die RSU die einzelnen Standorte aus und verschickt, sofern Gefahr für ein Fahrzeug besteht, eine ``Fußgänger-Warnung'' mittels dem Vehicle-to-everything (V2X) Kommunikationsstandard an das betroffene Fahrzeug.

\end{german}